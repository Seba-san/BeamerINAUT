\usepackage[utf8]{inputenc} %acentos sin codigo extra \usepackage[utf8]{inputenc}
\usepackage{graphicx}%es para agregar imagenes
%\usepackage{subfigure}  %es para las subfiguras
\usepackage{subcaption} % poner mejores subfiguras?
\usepackage[spanish,es-noshorthands,es-tabla]{babel} %agrega unos caracteres extra al spanish de babel
\usepackage{alltt} %Ver para que es esto, me parece que simplifica el "` vervatim"'
\usepackage{listings} % es para poder cargar los codigos y que se vean bonitos
\renewcommand\lstlistingname{Código} % Sirve para cambiar el titulo de los codigos. En vez de decir "listing"
%\usepackage[hidelinks]{hyperref} % para agregar links "`[hidelinks]"' sirve para sacar el recuadro alrededor del link
\usepackage{amsthm} % ambiente matematico
\usepackage{amsmath} %otro ambiente matematico
\usepackage{mathtools}%otro ambiente matematico
\usepackage{color} % para poner textos con color 
\usepackage{hyperref} % para poner links
\usepackage{epsfig} % para poner color en las tablas
\usepackage{multirow}% para poner color en las tablas
\usepackage{colortbl}% para poner color en las tablas. Esta es la mas importante 
%\usepackage[table]{xcolor}% para poner color en las tablas % http://ctan.org/pkg/xcolor
\usepackage{adjustbox} % Para que las tablas no se salgan de los margenes
\usepackage{fancyhdr} % Es para el encabezado y el pie de pagina
\usepackage{xcolor}% para poner color % http://ctan.org/pkg/xcolor
\usepackage{tikz}  % Para superponer imaganes
\usepackage{amssymb} % Para agregar simbolos distintos al itemize.
% Para justificar
\usepackage{ragged2e}
%\usepackage{etoolbox}
%\usepackage{lipsum}
% Para poner pseudocodigos
\usepackage{algorithm}
%\usepackage[noend]{algpseudocode}
\usepackage{algorithm,algorithmic}

\newcommand{\matlab}{MATLAB\textsuperscript{\textregistered}\ }
% Para poner un lindo codigo de matlab:
% For faster processing, load Matlab syntax for listings
\definecolor{codegreen}{rgb}{0,0.6,0}
\definecolor{codegray}{rgb}{0.5,0.5,0.5}
\definecolor{codepurple}{rgb}{0.58,0,0.82}
\definecolor{backcolour}{rgb}{0.95,0.95,0.92}
 
\lstdefinestyle{mystyle}{
    backgroundcolor=\color{backcolour},   
    commentstyle=\color{codegreen},
    keywordstyle=\color{magenta},
    numberstyle=\tiny\color{codegray},
    stringstyle=\color{codepurple},
    basicstyle=\ttfamily\footnotesize,
    breakatwhitespace=true,         
    breaklines=true,                 
    captionpos=b,                    
    keepspaces=false,                 
    numbers=left,                    
    numbersep=5pt,                  
    showspaces=false,                
    showstringspaces=false,
    showtabs=false,                  
    tabsize=2
}
 
\lstset{style=mystyle}
  %---------------------------------------------------------
